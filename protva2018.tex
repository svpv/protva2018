\documentclass[russian,a4paper,12pt]{article}
\usepackage[utf8]{inputenc}
\usepackage[russian]{babel}
\usepackage{url}
\usepackage{fullpage}
\usepackage{comment}

\sloppy
\hyphenpenalty=666

\begin{document}
\title{\texttt{zrec}~--- формат\\метаданных репозитория}
\author{Алексей Турбин}
\date{17 сентября 2018 г.}
\maketitle

\begin{abstract}

\end{abstract}

\section{Введение}
Перед обновлением пакетов из репозитория скачивается большой файл с \textit{записями пакетов} (records);
в \verb|apt-rpm| это файлы \verb|pkglist.xz| и \verb|srclist.xz|, содержащие урезанные
rpm-заголовки (в \verb|repomd|-репозиториях записями можно считать сегменты XML-файла).
Файл \verb|pkglist.xz| хорошо сжат, но распаковка его занимает несколько секунд, и далее
он хранится в \verb|/var/lib/apt| в разжатом виде, занимая довольно много места (200--300 Мб;
здесь не столько жалко места на диске, сколь возникают <<тормоза>> при доступе к файлу).
Пережать же в более <<легкий>> формат \verb|pkglist| нельзя, поскольку \verb|apt| требует произвольный
доступ к записям (команда \texttt{apt-cache show} делает \verb|lseek(2)| и считывает запись).

В 2017 г.~автор предпринял попытку разработать <<легкий>> формат \verb|zpkglist|
для сжатия записей после скачивания \cite{zpkglist}.
За основу была взята библиотека сжатия \verb|LZ4|; для нее была адаптирована техника сжатия со словарем,
реализованная в библиотеке \verb|zstd|.  Перед сжатием записи группировались по 4 штуки, что значительно
улучшало коэффициент сжатия, однако произвольный доступ требовал распаковки группы из четырех записей.

В мае 2018 г.~Джонатан Дайетер (Jonathan Dieter) анонсировал похожий проект \verb|zchunk| \cite{dieter}
(формат файла и библиотека сжатия).  Поддержка \verb|zchunk| добавлена в \verb|dnf|, планируется к использованию
в Fedora~29 либо Fedora~30.  \verb|zchunk| основывается на алгоритме \verb|zstd|, который способен, в отличие от \verb|LZ4|,
обеспечить значительно более высокий, <<релизный>> уровень сжатия (хотя и несколько не дотягивает до \verb|xz|).
Поэтому интерес автора стал смещаться в сторону использования нового формата для сжатия на сервере (сжатый файл
не требует распаковки на клиенте).

Файл в формате \verb|zchunk| можно понимать просто как сжатый поток байтов.  При сжатии поток нарезается на куски (chunking)
и каждый кусок сжимается хотя и отдельно, но \textit{со словарем}, отдельно хранящемся в файле (что значительно улучшает
коэффициент сжатия).  Естественно, при сжатии записей нарезка идет по границам записей, но в одном куске может содержаться
и несколько записей.  В начале же файла хранится \textit{индекс} кусков: их размеры и хеш-суммы.  Наличие индекса делает
возможным \textit{синхронизацию}: клиент сначала скачивает индекс и потом~--- через HTTP range requests~--- недостающие
куски, перестраивая старый файл в новый.

Формат \verb|zchunk| не устраивает автора лишь в деталях.  Далее рассмотрены некоторые специальные особенности сжатия
и синхронизации, которые приводят к созданию альтернативного формата~--- \verb|zrec| (сокр.~от compressed records).

\section{О пользе сортировки}
\section{Группировка записей}
\section{Хеширование и расстановка}
\section{О надежности конструкции}

\begin{thebibliography}{9}

\bibitem{zpkglist}
Alexey Tourbin. \verb|zpkglist|~--- Compressed file format\\
\url{https://github.com/svpv/zpkglist}

\bibitem{dieter}
Jonathan Dieter. What is \verb|zchunk|?\\
\url{https://www.jdieter.net/posts/2018/05/31/what-is-zchunk/}

\end{thebibliography}

\end{document}
